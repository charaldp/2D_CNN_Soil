\chapter{Εισαγωγή}
\label{ch:introduction}
\section{Σχετικά με το \tl{Soil Spectroscopy}}
Η ραγδαία ανάπτυξη της τεχνολογίας που παρατηρείται να επέρχεται όλο και περισσότερο στη ζωή μας, δύναται να προσφέρει απλοποιήσεις σε σύνθετες και κοστοβόρες διαδικασίες. Μια από τις περιπτώσεις όπου η τεχνολογία \textbf{έρχεται} για να δώσει μια αποτελεσματική λύση, αφορά την πιθανή χρήση της φασματοσκοπίας υπερύθρου (\tl{Soil Spectroscopy}).

Η χρήση μηχανισμών εξαγωγής φασματικών υπογραφών εδαφικής ύλης οδήγησε στην δημιουργία βιβλιοθηκών δεδομένων τα οποία αξιοποιούνται συχνά σε βιβλιογραφικές μελέτες και αναλύσεις. Ένα μέρος των σχετικών αναλύσεων δείχνουν πως με την χρήση μηχανικής μάθησης πάνω σε δεδομένα των συγκεκριμένων βιβλιοθηκών δύναται να εξαχθούν αξιόπιστες προβλέψεις σχετικά με τις χημικές ιδιότητες των εδαφικών δειγμάτων.

Το αντικείμενο της διπλωματικής εργασίας αφορά την επίλυση ενός προβλήματος παλινδρόμησης με τη χρήση συνελικτικών νευρωνικών δικτύων πάνω σε δεδομένα φασματοσκοπίας εδάφους (\tl{Soil Spectroscopy}).

Αρχικά οφείλεται να γίνει μια αναφορά στην τεχνολογία της φασματοσκοπίας εδάφους. Η συγκεκριμένη μέθοδος αποσκοπεί στην εξαγωγή χρήσιμης πληροφορίας σχετικά με εδαφικά δείγματα χρησιμοποιώντας την αλληλεπίδραση τους με ακτινοβολίες, έτσι εξετάζεται η αποφυγή μιας αναλυτικής μελέτης των χημικών ιδιοτήτων και συστατικών του εδάφους. Το αποτέλεσμα από την μέτρηση της ανακλώμενης ακτινοβολίας σε ένα εύρος μηκών κύματος αποτελεί ουσιαστικά την φασματική υπογραφή τους. Για την εξαγωγή των παραπάνω συνήθως χρησιμοποιούνται μήκη κύματος εύρους 350–2500 \tl{nm (Vis-near)} ή και 2500–25,000 \tl{nm (mid-infrared)}.

\section{Σχετικά με το έδαφος}
Το έδαφος αποτελεί ένα σημαντικό μέρος του πλανήτη μας, η χρησιμότητα του φαίνεται στις βασικές μας ανάγκες όπως η καλλιέργεια, .

Η κατάσταση του εδάφους σήμερα είναι σημαντικό να ελέγχεται συστηματικά καθώς έχουν παρατηρηθεί διάφορα σημάδια κίνδυνου τα οποία χρήζουν παρατήρησης. Κάποια από αυτά είναι η διάβρωση του εδάφους το οποίο οδηγεί στην επιδείνωση της ποιότητας του νερού αλλά και την ελάττωσή της σοδειάς η οποία δύναται να εξάγει μια εδαφική περιοχή. 

Είναι επιθυμητό να μπορούμε να ελέγξουμε την κατάσταση του εδάφους καθώς ανά περιοχές μπορεί να . Η σύσταση μιας εδαφικής ύλης μπορεί να μας δώσει πληροφορίες σχετικά με κάποιες χρήσιμες ιδιότητες του όπως οι .

Οι εργαστηριακές αναλύσεις εδαφικών δειγμάτων ήταν πάντα απαραίτητες για την καταγραφή των διακυμάνσεων των ιδιοτήτων. Η δειγματοληψία τους γίνεται όσο πυκνότερα είναι εφικτό να γίνει κάθε έκταση της γης, ώστε να καταγράφεται η ιδιομορφία κάθε περιοχής, ενώ ταυτόχρονα να είναι πρακτικά υλοποιήσιμη.

Ωστόσο πέρα από τις εργαστηριακές μετρήσεις δειγμάτων εδάφους προτιμάται μια μέθοδος ανάκτησης ιδιοτήτων εδάφους για μεγάλες εκτάσεις με μικρό κόστος

\section{Στόχοι της διπλωματικής εργασίας}
Οι στόχοι της διπλωματικής εργασίας είναι η μελέτη της μεθόδου κατασκευής δισδιάστατων νευρωνικών δικτύων για την εκτίμηση ιδιοτήτων εδάφους, δηλαδή ενός προβλήματος παλινδρόμησης. Κατά την μελέτη των ήδη υπαρχόντων υλοποιήσεων γίνεται μια προσπάθεια για την εύρεση μοντέλων τα οποία έχουν απλούστερη αρχιτεκτονική έτσι η εκπαίδευση τους είναι λιγότερο απαιτητική, ενώ ταυτόχρονα επιτεγχάνεται βελτίωση στην επίδοση του μοντέλου στην ικανότητα πρόβλεψης των ιδιοτήτων εδάφους. 

