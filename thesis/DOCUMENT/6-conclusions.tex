\chapter{Συμπεράσματα και μελλοντικές επεκτάσεις}
\label{ch:conclusions}
\section{Συμπεράσματα}
Στην παρούσα διπλωματική εργασία έγινε η ανάλυση της λειτουργίας των δισδιάστατων συνελικτικών νευρωνικών δικτύων και της εφαρμογής τους στο πρόβλημα της φασματοσκοπία εδάφους. Μετά από ανάλυση μιας προτεινόμενης υλοποίησης εξετάστηκαν μέθοδοι για την βελτίωση της και την εύρεση ενός βέλτιστου μοντέλου το οποίο θα είναι ικανό να παράγει αξιόλογες προβλέψεις με απλοποιημένη αρχιτεκτονική δομή. Όπως φαίνεται η επίτευξη του παραπάνω είναι εφικτή και οι επιδόσεις των μοντέλων στα οποία καταλήξαμε έχουν σημαντικά καλύτερες επιδόσεις από τα προτεινόμενα.

\section{Επεκτάσεις}
Ορισμένα θέματα που θα μπορούσαν να διερευνηθούν περαιτέρω σχετικά με το αντικείμενο της παρούσας διπλωματικής εργασίας είναι, η διαπίστωση της σημασίας των διαστάσεων της εικόνας και της συσχέτισης της με τις διαστάσεις των συνελικτικών φίλτρων αλλά και την παράμετρο της υποδειγματοληψίας των φασματικών υπογραφών. Επίσης θα μπορούσε να βρεθεί κάποιος διαφορετικός μετασχηματισμός ή μέθοδος για την αναπαράσταση της φασματικής υπογραφής σε δισδιάστατη μορφή, με σκοπό την αποτύπωση κάποιας επιπλέον χρήσιμης πληροφορίας για το σήμα εισόδου.