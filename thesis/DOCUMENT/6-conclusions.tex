\chapter{Συμπεράσματα και μελλοντικές επεκτάσεις}
\label{ch:conclusions}
\section{Συμπεράσματα}
Στην παρούσα διπλωματική εργασία έγινε η ανάλυση της λειτουργίας των δισδιάστατων συνελικτικών νευρωνικών δικτύων και της εφαρμογής τους στο πρόβλημα της φασματοσκοπίας εδάφους στο εργαστήριο. Μετά από ανάλυση μιας προτεινόμενης υλοποίησης εξετάστηκαν μέθοδοι για την βελτίωση της και την εύρεση ενός βέλτιστου μοντέλου το οποίο θα είναι ικανό να παράγει αξιόλογες προβλέψεις με απλοποιημένη αρχιτεκτονική δομή.\\
Οι τροποποιήσεις που έγιναν είναι η μείωση του πλήθους των δισδιάστατων συνελικτικών επιπέδων και η χρήση περισσοτέρων πλήρως συνδεδεμένων επιπέδων όσων αφορά την αρχιτεκτονική του μοντέλου. Ταυτόχρονα βελτιστοποιήθηκαν παράμετροι όπως ο λόγος της κατακόρυφης προς την οριζόντια διάσταση της εικόνας εισόδου και η δειγματοληψία που εφαρμόζεται στις φασματικές υπογραφές πριν την μετατροπή τους σε σπεκτρογράμματα. Επιπλέον έγινε η χρήση του βελτιστοποίητη \tl{Nadam} ο οποίος φάνηκε πως εκτελούσε με τον πιο αποτελεσματικό τρόπο την διαδικασία της εκπαίδευσης. Τέλος, χρησιμοποιήθηκε η διαφορετική μορφή του σπεκτρογράμματος με χρήση του μετασχηματισμού \tl{Savitzky-Golay} 1ης παραγώγου της απορροφητικότητας σε συνδυασμό με τις φασματικές υπογραφές ανακλαστικότητας, έτσι το τελικό μοντέλο είχε 2 εισόδους.\\
Όπως φάνηκε οι στόχοι της παρούσας διπλωματικής εργασίας επιτεύχθησαν διότι οι επιδόσεις των μοντέλων στα οποία καταλήξαμε έχουν σημαντικά καλύτερες επιδόσεις από τα προτεινόμενα.

\section{Επεκτάσεις}
Ορισμένα θέματα που θα μπορούσαν να διερευνηθούν περαιτέρω σχετικά με το αντικείμενο της παρούσας διπλωματικής εργασίας είναι η διαπίστωση της σημασίας των διαστάσεων της εικόνας και της συσχέτισης της με τις διαστάσεις των συνελικτικών φίλτρων αλλά και την παράμετρο της υποδειγματοληψίας των φασματικών υπογραφών. Επίσης, θα μπορούσε να βρεθεί κάποιος διαφορετικός μετασχηματισμός ή μέθοδος για την αναπαράσταση της φασματικής υπογραφής σε δισδιάστατη μορφή, με σκοπό την αποτύπωση κάποιας επιπλέον χρήσιμης πληροφορίας για το σήμα εισόδου. Τέλος, στην ανάλυση που έγινε θα μπορούσε να διερευνηθεί η χρήση των παραμέτρων γεωγραφικής γειτνίασης των δειγμάτων εδάφους, με σκοπό την βελτίωση της επίδοσης των μοντέλων.