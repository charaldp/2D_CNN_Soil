\chapter{Η εδαφική βάση \tl{LUCAS-Soil} }

\section{Σχετικά με το σετ δεδομένων}
Η βάση εδαφικών δειγμάτων \tl{LUCAS-Soil (Land Use/Cover Area frame statistical Survey Soil)} αποτελεί μια από τις μεγαλύτερες και διαρκώς αναπτυσσόμενες βάσεις εδαφικών δεδομένων. Τα δείγματα της έχουν συγκεντρωθεί από διάφορες περιοχές της Ευρώπης. 
Τα δεδομένα της συγκεκριμένης βάσης χωρίονται σε 2 βασικές κατηγορίες, τα δείγματα \tl{Mineral} και \tl{Organic}. Τα δείγματα της κατηγορίας \tl{Mineral} αποτελούνται από 3 κύριες υποκατηγορίες εδάφους \tl{Grassland, Woodland, Cropland}. Η ποικιλία στις κατηγορίες των εδαφικών δειγμάτων προσφέρει ένος εύρος των χαρακτηριστικών τους όσων αφορά την ορυκτολογία, την υφή, τον σίδηρο και την περιεκτικότητά τους σε ανθρακικό ασβέστιο \cite{nocita_lucas_soil}. Αυτό καθιστά την εδαφική βάση ικανή για την εκτέλεση πειραμάτων τα οποία μπορούν να πραγματοποιούνται προβλέψεις ιδιοτήτων γενικευμένων ειδών εδάφους.

Τα δεδομένα εδάφους που χρησιμοποιούνται στην παρούσα διπλωματική εργασία έχουν υποστεί μια επεξεργασία, καθώς ορισμένα από τα δείγματα του σετ δεδομένων τα οποία έχουν εξωκείμενες τιμές για βασικές ιδιότητες τους όπως ο οργανικός άνθρακας (~200 - 586.8 $g~C~kg^{-1}$), έχουν αφαιρεθεί καθώς η συμβολή τους κατά την εκπαίδευση μοντέλων και την εξαγωγή μετρικών είναι μεγάλη, έτσι δεν είναι εφικτό να εξαχθούν βάσιμα συμπεράσματα σχετικά με τα δεδομένα. Το τελικό πλήθος δειγμάτων που χρησιμοποιούνται είναι περίπου 18000 από τα συνολικά 19000. Τα δείγματα που αφαιρούνται ανήκουν κυρίως στην κατηγορία \tl{Organic}. Ένας ακόμη λόγος που δεν χρησιμοποιείται η συγκεκριμένη κατηγορία είναι πως δεν περιέχει δεδομένα σχετικά με την υφή του εδάφους.

\section{Υπάρχουσες τεχνικές}
Στην υπάρχουσα βιβλιογραφία φαίνεται να έχουν δοκιμαστεί ορισμένες υλοποιήσεις για την ανάλυση των δεδομένων της εδαφικής βάσης του \tl{LUCAS}. Οι υλοποιήσεις που εφαρμόστηκαν εξετάζουν διάφορα είδη μοντέλων μηχανικής μάθησης για παλινδρόμηση. Επίσης αναλύονται οι επιμέρους υποκατηγορίες των δειγμάτων εδάφους ενώ δοκιμάζονται ορισμένες τεχνικές προ-επεξεργασίας των δεδομένων εισόδου ή εφαρμογής μετασχηματισμών σε αυτά.

\subsection{Δημοσίευση του \tl{Stevens A.}}
Στη δημοσίευση του \tl{Stevens A.} \cite{stevens_lucas_soil} αναλύονται τα εδαφικά δείγματα στις κατηγορίες \tl{Mineral} και \tl{Organic} καθώς και τα επιμέρους μέρη της κατηγορίας \tl{Mineral} τα οποία είναι τα \tl{Grassland, Woodland} και \tl{Cropland}. Το μοντέλο πρόβλεψης που χρησιμοποιείται για την παλινδρόμηση είναι ένα πολυμεταβλητό μοντέλο. Παρουσιάζονται όλα τα στατιστικά στοιχεία για όλες της ιδιότητες εδάφους της εδαφικής βάσης του \tl{LUCAS}, ενώ η περαιτέρω ανάλυση εστιάζει στην πρόβλεψη της περιεκτικότητας σε οργανικό άνθρακα καθώς θεωρείται το χρησιμότερο στοιχείο.

Στην ανάλυση που πραγματοποιείται εξετάζεται επίσης η χρήση της περιεκτικότητας σε άμμο σαν εξωτερικός προβλέπτης και γίνεται σύγκριση της επίδοσης με την χρήση μόνο φασματικών υπογραφών ή και με την χρήση του εξωτερικού προβλέπτη. Για την επιλογή των βέλτιστων μοντέλων πρόβλεψης ενδείκνυνται διάφορες μεθόδοι προ-επεξεργασίας των δεδομένων εισόδου όπως ο μετασχηματισμός εξομάλυνσης των \tl{Savitzky}--\tl{Golay} \cite{savitzky_golay}, 1η παράγωγος αυτού του μετασχηματισμού αλλά και ο μετασχηματισμός τυπικής κανονικοποιημένης μεταβλητότητας --- \tl{SNV (Standard Normal Variate)}. Η πρόβλεψη των μοντέλων εξετάζεται σε διαφορετικά μέρη των τμημάτων του σετ δεδομένων τα οποία έχουν ομαδοποιηθεί με βάση την περιεκτικότητα τους σε οργανικό άνθρακα.

\subsection{Δημοσίευση του \tl{Nocita M.}}
Μια παρόμοια ανάλυση γίνεται στο άρθρο του \tl{Nocita M.} \cite{nocita_lucas_soil}, όπου αποπειράται η πρόβλεψη του περιεχομένου οργανικού άνθρακα με τη χρήση μοντέλου μερικών ελαχίστων τετραγώνων. Με τη συγκεκριμένη μέθοδο η πρόβλεψη του μοντέλου προκύπτει με τη χρήση των τιμών ενός πλήθους των κοντινότερων γειτόνων---\tl{nearest neighbours}, οπού το κριτήριο γειτνίασης αφορά την ομοιότητα σε στοιχεία του φάσματος του εκαστοτε εδαφικού δείγματος.

Οι προβλέψεις των μοντέλων χωρίζονται σε 4 διαφορετικά τμήματα της εδαφικής βάσης του \tl{LUCAS}, τα \tl{Grassland, Woodland, Cropland} και \tl{Organic}. Έπειτα από την δοκιμή διαφόρων μεγεθών του αριθμού κοντινότερων γειτόνων προκύπτει η βέλτιστη επίδοση με τη χρήση ενός πλήθους 250 γειτόνων για τα μοντέλα κάθε διαφορετικής κατηγορίας εδάφους.

\subsection{Δημοσίευση του \tl{Padarian J.}}
Μια πιο πρόσφατη προσέγγιση η οποία εξετάζει διεξοδικά τις σημαντικότερες από τις ιδιότητες του εδάφους και είναι το επίκεντρο της παρούσας διπλωματικής εργασίας, είναι η δημοσίευση του \tl{Padarian J.} \cite{padarian_lucas_soil}. Στη συγκεκριμένη μέθοδο γίνεται η χρήση δισδιάστατου νευρωνικού δικτύου, όπου το δισδιάστατο σήμα εισόδου προκύπτει από το μετασχηματισμό \tl{wavelet} του φάσματος ανακλαστικότητας και αποτελεί ουσιαστικά ένα σπεκτρόγραμμα του. Το μοντέλο που προτείνεται αποτελείται από πολλαπλά συνελικτικά επίπεδα, καθώς και επίπεδα συγκέντρωσης μεγίστων και πυκνά συνδεδεμένων επιπέδων.

Η πρόβλεψη των ιδιοτήτων εδάφους της βάσης του \tl{LUCAS} εστιάζει στην περιεκτικότητα οργανικού άνθρακα, την ικανότητα ανταλλαγής κατιόντων, περιεκτικότητα σε πηλό, άμμο, το \tl{pH} και το άζωτο. Για κάθε μία από τις 6 αυτές ιδιότητες δημιουργείται ένα μοντέλο μιας εξόδου, παράλληλα εξετάζεται η επίδοση ενός μοντέλου με παρόμοια αρχιτεκτονική το οποίο όμως έχει πολλές εξόδους. Η σύγκρισή των 2 παραπάνω ειδών μοντέλων γίνεται μαζί με μοντέλα μερικών ελαχίστων τετραγώνων και \tl{Cubist} με καλύτερο από αυτά το μοντέλο πολλών εξόδων. Ωστόσο παρατηρείται πως η επίδοση του μοντέλου πολλών εξόδων απαιτεί ένα ένα επαρκές μέγεθος για το σετ δεδομένων εκπαίδευσης έτσι η επίδοση του είναι χειρότερη σε μια δοκιμή με χρήση ενός μικρότερου σετ δεδομένων.
