\chapter{Η εδαφική βάση \tl{LUCAS-Soil} }
\label{ch:lucas_soil}

\section{Σχετικά με το σετ δεδομένων}
Η βάση εδαφικών δειγμάτων \tl{LUCAS-Soil (Land Use/Cover Area frame statistical Survey Soil)} αποτελεί μια από τις μεγαλύτερες και διαρκώς αναπτυσσόμενες βάσεις εδαφικών δεδομένων. Τα δείγματα της έχουν συγκεντρωθεί από διάφορες περιοχές της Ευρώπης. 
Τα δεδομένα της συγκεκριμένης βάσης δύναται να χωριστούν σε 2 βασικές κατηγορίες, τα ορυκτά (\tl{\textbf{Mineral}}) δείγματα και οργανικά (\tl{Organic}) δείγματα, βάσει του ορισμού του \tl{FAO} για οργανικά εδάφη. Τα δείγματα της κατηγορίας \tl{Mineral} αποτελούνται από 3 κύριες χρήσεις γης: την χορτολιβαδική, τη δασική και την καλλιεργήσιμη \tl{(\textbf{Grassland}, \textbf{Woodland}, \textbf{Cropland})}, αντίστοιχα. Η ποικιλία στις κατηγορίες των εδαφικών δειγμάτων προσφέρει ένα εύρος των χαρακτηριστικών τους όσων αφορά την ορυκτολογία, την υφή, τον σίδηρο και την περιεκτικότητά τους σε ανθρακικό ασβέστιο \cite{nocita_lucas_soil}. Αυτό καθιστά την εδαφική βάση ικανή για την εκτέλεση πειραμάτων τα οποία μπορούν να πραγματοποιούνται προβλέψεις ιδιοτήτων γενικευμένων ειδών εδάφους.

Στα δεδομένα της εδαφικής βάσης του \tl{LUCAS} παρατηρείται πως μια συγκεκριμένη κατηγορία εδαφικών δειγμάτων, και συγκεκριμένα τα οργανικά εδάφη, έχουν για την ιδιότητα της περιεκτικότητας σε οργανικό άνθρακα (~200 -- 586.8 $g~C~kg^{-1}$), ένα αρκετά μεγάλο εύρος δηλαδή σε σχέση με όλα τα υπόλοιπα εδαφικά δείγματα που έχουν τιμές (~0 -- 200 $g~C~kg^{-1}$). Έτσι τα συγκεκριμένα δείγματα έχουν αφαιρεθεί καθώς η συμβολή τους κατά την εκπαίδευση μοντέλων και την εξαγωγή μετρικών είναι μεγάλη, με τρόπου που δεν είναι εφικτό να εξαχθούν βάσιμα συμπεράσματα σχετικά με τα δεδομένα. Σε κάθε περίπτωση αυτό δεν αποτελεί τροχοπέδη στην εφαρμογή της μεθοδολογίας, διότι τα οργανικά εδάφη ξεχωρίζουν συνήθως εύκολα από τα ορυκτά. Το τελικό πλήθος δειγμάτων που χρησιμοποιούνται είναι περίπου 18000 από τα συνολικά 19000. Ένας ακόμη λόγος που δεν χρησιμοποιείται η συγκεκριμένη κατηγορία είναι πως δεν περιέχει δεδομένα σχετικά με την υφή του εδάφους.

\section{Ιδιότητες και φασματικές υπογραφές}
\label{sec:props-spectrograms}
Το αντικείμενο της διπλωματικής αφορά ένα θέμα που έχει ερευνητικό ενδιαφέρον λόγω της σημασίας της ικανότητας πρόβλεψης των διαφόρων ιδιοτήτων εδάφους. Σε αυτή την υποενότητα παρουσιάζονται οι χημικές ιδιότητες και η σημαντικότητα τους για το έδαφος αλλά και τα φυτά ή τις καλλιέργειες που αναπτύσσονται σε αυτό.

\begin{outline}
    \1 Οργανικός Άνθρακας (\tl{Soil Organic Carbon \textbf{OC}}): Ο συνολικός οργανικός άνθρακας αποτελεί μια από τις σημαντικότερες ιδιότητες του εδάφους καθώς είναι ένας δείκτης ποιότητας του εδάφους. Η συγκεκριμένη ιδιότητα επηρεάζει άμεσα την ικανότητα ανταλλαγής κατιόντων και την δυνατότητα συγκράτησης θρεπτικών συστατικών. Επίσης οργανικός άνθρακας διασπάται από ορισμένους χρήσιμους μικροοργανισμούς που περιέχονται στο έδαφος και αποτελεί ουσιαστικά το καύσιμο τους για την παραγωγή ενέργειας.
    \1 Ολικό άζωτο (\tl{Nitrogen \textbf{N}}):  Η ανάγκη μέτρησης του αζώτου στο έδαφος ανάγεται στο γεγονός ότι είναι η επί το πλείστων απαραίτητη ουσία για την ανάπτυξη των καλλιεργειών που βρίσκονται στο έδαφος, όντας βασικό θρεπτικό συστατικό.
    \1 Ικανότητα ανταλλαγής κατιόντων \tl{(Cation exchange capacity \textbf{CEC})}: Η ικανότητα ανταλλαγής κατιόντων του εδάφους δείχνει ουσιαστικά την δυνατότητα του να κρατά θρεπτικές ουσίες σε μορφή ανταλλάξιμων κατιόντων. Η ιδιότητα αυτή επηρεάζει άμεσα την συχνότητα και την ποσότητα πρόσθεσης κατιόντων στο εδαφών κατά την γονιμοποίηση του. Το κύριο κατιόν που ανταλλάσσεται από ένα γόνιμο έδαφος είναι το κάλιο ($\mathbf{K^+}$) ενώ υπάρχουν και τα εξής δευτερεύοντα, όπως το ασβέστιο ($\mathbf{Ca^{++}}$), το μαγνήσιο ($\mathbf{Mg^{++}}$) και το νάτριο ($\mathbf{Na^+}$), τα οποία μπορούν να είναι επαρκή ακόμη και όταν υπάρχουν σε μικρές ποσότητες.
    \1 Ανθρακικό Ασβέστιο (\tl{Calcium Carbonate} $\mathbf{CaCO_3}$): Η παρουσία του ανθρακικού ασβεστίου σε μια εδαφική ύλη έχει χρησιμότητα ως προς την δομική συνοχή της. Ένα εδαφικό δείγμα με αυξημένη περιεκτικότητα σε ανθρακικό ασβέστιο συνήθως θα έχει αυξημένη τιμή στο \tl{pH} του, θα είναι δηλαδή περισσότερο αλκαλικό.
    \1 \tl{\textbf{pH}} του νερού στο έδαφος: Η μέτρηση του \tl{pH} του νερού του εδάφους είναι χρήσιμη καθώς πρέπει να βρίσκεται εντός συγκεκριμένων ορίων ώστε να αναπτυχθεί ένα φυτό σε αυτό. Όταν το \tl{pH} του εδάφους όξινο $(\mathbf{pH<4.5})$ ή αλκαλικό $(\mathbf{pH>8.5})$ τα φυτά αναπτύσσονται δύσκολα σε αυτό, αν όχι καθόλου.
    \1 Άμμος, ιλύς και άργιλος (\tl{\textbf{Sand}, \textbf{Silt} and \textbf{Clay}}): Η άμμος, η ιλύς και ο άργιλος είναι χαρακτηριστικά που ορίζουν την υφή του εδάφους. Τα χαρακτηριστικά αυτά παρέχονται από την εδαφική βάση του \tl{LUCAS} σε ποσοστά επί τις \% έτσι ώστε $Sand+Silt+Clay=100$. Μια τριάδα ποσοστών περιεκτικότητας των 3 υλικών δίνει έναν χαρακτηρισμό για την υφή του συγκεκριμένου εδαφικού δείγματος. Το φυσικό μέγεθος στο οποίο παρατηρείται καθένα από τα 3 υλικά είναι $2-0.05mm$ για την άμμο, $0.05-0.002mm$ για την ιλύ και μικρότερο από $0.002mm$ για τον άργιλο.
        \2 Άμμος \tl{\textbf{Sand}}: Η ύπαρξη της άμμου στο έδαφος το καθιστά πορώδες έτσι παρέχει εξαερισμό στο έδαφος. Αυτό είναι σημαντικό καθώς με αυτό τον τρόπο επιβιώνουν χρήσιμα για το έδαφος μικρόβια. Η άμμος επίσης προσφέρει κάποια συνοχή στο έδαφος.
        \2 Ιλύς \tl{\textbf{Silt}}: Η περιεκτικότητα τους εδάφους σε ιλύ δίνει την δυνατότητα να συγκρατεί νερό και λιπασματικά στοιχεία γονιμοποίησης του. Η μεγάλη περιεκτικότητα σε ιλύ μπορεί να οδηγήσει σε δυσκολία ανάπτυξης ορισμένων καλλιεργειών στα τμήματα τους που βρίσκονται εσωτερικά του εδάφους.
        \2 Άργιλος \tl{\textbf{Clay}}: Ο άργιλος είναι μια ενδιάμεση μορφή της άμμου και της ιλύος προσφέροντας το πλεονέκτημα του εξαερισμού στο έδαφος και της συγκράτησης νερού, σε κάποιο ενδιάμεσο βαθμό.
    \1 Φώσφορος (\tl{Phosphorus \textbf{P}}): Η σημασία του φωσφόρου στο έδαφος είναι φαίνεται μέσω της θετικής επίδρασης στα φυτά που βρίσκονται σε αυτό στην διαδικασία της φωτοσύνθεσης αλλά και στην καρποφορία τους. Επίσης ο φώσφορος φαίνεται πως βοηθάει στην ανάπτυξη των ριζών φυτών.
    \1 Κάλιο (\tl{Potassium \textbf{K}}): Η περιεκτικότητα του καλίου στο έδαφος έχει παρατηρηθεί πως επηρεάζει σε ένα βαθμό την ανάπτυξη μιας καλλιέργειας αλλά και την ποιότητα της.
\end{outline}
Μια στατιστική ανάλυση των ιδιοτήτων των δειγμάτων της εδαφικής βάσης του \tl{LUCAS} μπορεί να αναδείξει κατά πόσο είναι εφικτή η πρόβλεψη κάθε ιδιότητας, καθώς και την την αξιοπιστία ορισμένων μετρικών που θα προκύψουν κατά την αξιολόγηση των μοντέλων. Για παράδειγμα όταν υπάρχει μεγάλη τυπική απόκλιση σε μία ιδιότητα αυτό μπορεί να έχει ως αποτέλεσμα την εξαγωγή μιας αξιόλογης τιμής του συντελεστή προσδιορισμού $R^2$ ενώ η τιμή του μέσου τετραγωνικού σφάλματος να είναι σχετικά μεγάλη. Στα στατιστικά μεγέθη του παρακάτω πίνακα επίσης φαίνεται πως η ασυμμετρία (\tl{\textbf{Skewness}}) είναι πιθανό να συσχετίζεται δυσκολίας πρόβλεψης μιας ιδιότητας, αυτό διαπιστώνεται για τις τιμές του καλίου $\mathbf{K}$ και του φωσφόρου $\mathbf{P}$ όπου είναι αισθητά μεγαλύτερες από τις τιμές ασυμμετρίας των υπολοίπων ιδιοτήτων. \\

\begin{table*}\centering
    \caption{Πίνακας στατιστικών των ιδιοτήτων εδάφους των 17937 ορυκτών δειγμάτων της εδαφικής βάσης του \tl{LUCAS}}
    \ra{1.3}\selectlanguage{english}
    \begin{tabular}{@{}rrrrrrrrr@{}}\toprule
        \tg{Ιδιότητα}&Min.&Q25&Q50&Q75&Max.&Mean&St. Dev.&Skew.\\
        \midrule
        OC ($g~kg^{-1}$)       & 0.0  & 12.3 & 19.6  & 34.7  & 199.2  & 29.35  & 28.85  & 2.67\\
        N ($g~kg^{-1}$)        & 0.0  & 1.2  & 1.7   & 2.6   & 16.2   & 2.15   & 1.61   & 2.44\\
        Clay ($\%$)            & 0.0  & 8.0  & 17.0  & 27.0  & 79.0   & 18.88  & 13.00  & 0.91\\
        Sand ($\%$)            & 1.0  & 19.0 & 42.0  & 64.0  & 99.0   & 42.88  & 26.10  & 0.18\\
        Silt ($\%$)            & 0.0  & 25.0 & 37.0  & 51.0  & 92.0   & 38.22  & 18.29  & 0.20\\
        CEC ($cmol^+kg^{-1}$)  & 0.0  & 6.8  & 11.7  & 18.7  & 137.0  & 14.14  & 10.54  & 1.94\\
        pH                     & 3.41 & 5.18 & 6.34  & 7.53  & 10.08  & 6.30   & 1.31   & -0.13\\
        K ($mg~kg^{-1}$)       & 0.0  & 69.2 & 131.9 & 239.7 & 7342.0 & 191.32 & 228.92 & 8.93\\
        CaCO$_3$ ($g~kg^{-1}$) & 0.0  & 0.0  & 1.0   & 16.0  & 944.0  & 54.58  & 128.40 & 2.87\\
        P ($mg~kg^{-1}$)       & 0.0  & 10.7 & 21.8  & 42.4  & 1366.4 & 29.56  & 32.68  & 6.67\\
        \bottomrule
    \end{tabular}\selectlanguage{greek}
\end{table*}

\begin{table*}\centering
    \caption{Πίνακας στατιστικών των ιδιοτήτων εδάφους της των 1102 δειγμάτων κατηγορίας οργανικών της εδαφικής βάσης του \tl{LUCAS}}
    \ra{1.3}\selectlanguage{english}
    \begin{tabular}{@{}rrrrrrrrr@{}}\toprule
        \tg{Ιδιότητα}&Min.&Q25&Q50&Q75&Max.&Mean&St. Dev.&Skew.\\
        \midrule
        OC ($g~kg^{-1}$)       & 91.1 & 297.2   & 401.1 & 474.94 & 586.8  & 386.79 & 101.53 & -0.26\\
        N ($g~kg^{-1}$)        & 3.1  & 11.2    & 14.5  & 19.0   & 38.6   & 15.52  & 5.74   & 0.71\\
        CEC ($cmol^+kg^{-1}$)  & 0.0  & 23.77   & 31.8  & 42.5   & 234.0  & 41.95  & 33.00  & 2.53\\
        pH                     & 3.21 & 3.98    & 4.26  & 4.74   & 7.49   & 4.46   & 0.72   & 1.35\\
        K ($mg~kg^{-1}$)       & 0.0  & 138.625 & 238.8 & 381.55 & 1884.9 & 290.17 & 214.60 & 2.00\\
        CaCO$_3$ ($g~kg^{-1}$) & 0.0  & 0.0     & 0.0   & 1.0    & 418.0  & 2.91   & 19.90  & 14.65\\
        P ($mg~kg^{-1}$)       & 0.0  & 16.4    & 30.5  & 50.925 & 259.2  & 38.36  & 34.45  & 2.04\\
        \bottomrule
    \end{tabular}\selectlanguage{greek}
\end{table*}

Η διαδικασία εξαγωγής του φάσματος διάχυτης ανάκλασης ή απορρόφησης ενός εδαφικού δείγματος προκύπτει από την παρακάτω διαδικασία. Εκπέμπεται ακτινοβολία σε ένα εύρος μηκών κύματος, (συνήθως $\left(400nm -
2500nm\right)$) προς το δείγμα, με αποτέλεσμα σε κάθε διαφορετική τιμή του μήκους κύματος να υπάρχει συγκεκριμένη ταλάντωση των μορίων του εδαφικού δείγματος. Η ταλάντωση προκαλεί απορρόφηση της ακτινοβολούμενης ισχύος στο δείγμα η οποία είναι αισθητή κατά την μέτρηση της ανακλώμενης από το έδαφος ακτινοβολίας. Έτσι προκύπτει ένα διάγραμμα της ανάκλασης ή απορρόφησης για πολλές τιμές του μήκους κύματος. Το διάγραμμα αυτό αναφέρεται ως φασματική υπογραφή του εδαφικού δείγματος, καθώς είναι σχεδόν μοναδική χαρακτηριστική μορφή για ένα εδαφικό δείγμα με συγκεκριμένες φυσικές και χημικές ιδιότητες, κάτι πάνω στο οποίο βασίζεται μεγάλο μέρος της ανάλυσης της φασματοσκοπίας εδάφους.

\begin{figure}[H]
  \begin{center}
    \includesvg[width=1\textwidth]{LUCAS_SOIL/Reflectances}
    \caption{Μορφή των φασματικών υπογραφών ανακλαστικότητας \tl{Reflectance}} 6 τυχαίων δειγμάτων του σετ δεδομένων
  \end{center}
\end{figure}

\section{Υπάρχουσες τεχνικές}
Στην υπάρχουσα βιβλιογραφία φαίνεται να έχουν δοκιμαστεί ορισμένες υλοποιήσεις για την ανάλυση των δεδομένων της εδαφικής βάσης του \tl{LUCAS}. Οι υλοποιήσεις που εφαρμόστηκαν εξετάζουν διάφορα είδη μοντέλων μηχανικής μάθησης για παλινδρόμηση. Επίσης αναλύονται οι επιμέρους υποκατηγορίες των δειγμάτων εδάφους ενώ δοκιμάζονται ορισμένες τεχνικές προ-επεξεργασίας των δεδομένων εισόδου ή εφαρμογής μετασχηματισμών σε αυτά.

\subsection{Δημοσίευση των \tl{Stevens et al.}}
Στη δημοσίευση του \tl{Stevens A.} \cite{stevens_lucas_soil} αναλύονται τα εδαφικά δείγματα στις κατηγορίες \tl{Mineral} και \tl{Organic} καθώς και τα επιμέρους μέρη της κατηγορίας \tl{Mineral} τα οποία είναι τα \tl{Grassland, Woodland} και \tl{Cropland}. Το μοντέλο πρόβλεψης που χρησιμοποιείται για την παλινδρόμηση είναι ένα πολυμεταβλητό μοντέλο. Παρουσιάζονται όλα τα στατιστικά στοιχεία για όλες της ιδιότητες εδάφους της εδαφικής βάσης του \tl{LUCAS}, ενώ η περαιτέρω ανάλυση εστιάζει στην πρόβλεψη της περιεκτικότητας σε οργανικό άνθρακα καθώς θεωρείται το χρησιμότερο στοιχείο.

Στην ανάλυση που πραγματοποιείται εξετάζεται επίσης η χρήση της περιεκτικότητας σε άμμο σαν επιπλέον χαρακτηριστικό εισόδου και γίνεται σύγκριση της επίδοσης με την χρήση μόνο φασματικών υπογραφών ή και με την χρήση του επιπλέον χαρακτηριστικού εισόδου. Για την επιλογή των βέλτιστων μοντέλων πρόβλεψης ενδείκνυνται διάφορες μεθόδοι προ-επεξεργασίας των δεδομένων εισόδου όπως ο μετασχηματισμός εξομάλυνσης των \tl{Savitzky}--\tl{Golay} \cite{savitzky_golay}, 1η παράγωγος αυτού του μετασχηματισμού αλλά και ο μετασχηματισμός τυπικής κανονικοποιημένης μεταβλητότητας --- \tl{SNV (Standard Normal Variate)}. Η πρόβλεψη των μοντέλων εξετάζεται σε διαφορετικά μέρη των τμημάτων του σετ δεδομένων τα οποία έχουν ομαδοποιηθεί με βάση την περιεκτικότητα τους σε οργανικό άνθρακα.

\subsection{Δημοσίευση των \tl{Nocita et al.}}
Μια παρόμοια ανάλυση γίνεται στο άρθρο των \tl{Nocita et al.} \cite{nocita_lucas_soil}, όπου αποπειράται η πρόβλεψη του περιεχομένου οργανικού άνθρακα με τη χρήση μοντέλου μερικών ελαχίστων τετραγώνων. Με τη συγκεκριμένη μέθοδο η πρόβλεψη του μοντέλου προκύπτει με τη χρήση των τιμών ενός πλήθους των κοντινότερων γειτόνων---\tl{nearest neighbours}, οπού το κριτήριο γειτνίασης αφορά την ομοιότητα σε στοιχεία του φάσματος του εκάστοτε εδαφικού δείγματος, επίσης ελέγχεται και η επίδοση με χρήση της γεωγραφικής γειτνίασης των εδαφικών δειγμάτων αλλά και η περιεκτικότητα σε άμμο ως χαρακτηριστικά για την πρόβλεψη.

Οι προβλέψεις των μοντέλων χωρίζονται σε 4 διαφορετικά τμήματα της εδαφικής βάσης του \tl{LUCAS}, με βάση τις 3 χρήσεις γης \tl{Grassland, Woodland, Cropland} και την κατηγορία εδαφικών δειγμάτων \tl{Organic}. Έπειτα από την δοκιμή διαφόρων μεγεθών του αριθμού κοντινότερων γειτόνων προκύπτει η βέλτιστη επίδοση με τη χρήση ενός πλήθους 250 γειτόνων για τα μοντέλα κάθε διαφορετικής κατηγορίας εδάφους.

\subsection{Δημοσίευση των \tl{Padarian et al.}}
Μια πιο πρόσφατη προσέγγιση η οποία εξετάζει διεξοδικά τις σημαντικότερες από τις ιδιότητες του εδάφους και είναι το επίκεντρο της παρούσας διπλωματικής εργασίας, είναι η δημοσίευση των \tl{Padarian et al.} \cite{padarian_lucas_soil}. Στη συγκεκριμένη μέθοδο γίνεται η χρήση δισδιάστατου νευρωνικού δικτύου, όπου το δισδιάστατο σήμα εισόδου προκύπτει από το μετασχηματισμό \tl{wavelet} του φάσματος ανακλαστικότητας και αποτελεί ουσιαστικά ένα σπεκτρόγραμμα του. Το μοντέλο που προτείνεται αποτελείται από πολλαπλά συνελικτικά επίπεδα, καθώς και επίπεδα συγκέντρωσης μεγίστων και πυκνά συνδεδεμένων επιπέδων.

Η πρόβλεψη των ιδιοτήτων εδάφους της βάσης του \tl{LUCAS} εστιάζει στην περιεκτικότητα οργανικού άνθρακα, την ικανότητα ανταλλαγής κατιόντων, περιεκτικότητα σε πηλό, άμμο, το \tl{pH} και το άζωτο. Για κάθε μία από τις 6 αυτές ιδιότητες δημιουργείται ένα μοντέλο μιας εξόδου, παράλληλα εξετάζεται η επίδοση ενός μοντέλου με παρόμοια αρχιτεκτονική το οποίο όμως έχει πολλές εξόδους. Η σύγκρισή των 2 παραπάνω ειδών μοντέλων γίνεται μαζί με μοντέλα μερικών ελαχίστων τετραγώνων και \tl{Cubist} με καλύτερο από αυτά το μοντέλο πολλών εξόδων. Ωστόσο παρατηρείται πως η επίδοση του μοντέλου πολλών εξόδων απαιτεί ένα ένα επαρκές μέγεθος για το σετ δεδομένων εκπαίδευσης έτσι η επίδοση του είναι χειρότερη σε μια δοκιμή με χρήση ενός μικρότερου σετ δεδομένων.
