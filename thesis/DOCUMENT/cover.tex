\begin{titlepage}

\begin{figure}[H]
  \begin{center}
    \includegraphics[width=3cm]{auth.pdf}
    \label{fig:cover_auth_logo}
  \end{center}
\end{figure}

\centering
\Large Αριστοτέλειο Πανεπιστήμιο Θεσσαλονίκης\\
\Large Πολυτεχνική Σχολή\\
\large Τμήμα Ηλεκτρολόγων Μηχανικών και Μηχανικών Υπολογιστών\\
\large Τομέας Ηλεκτρονικής και Υπολογιστών

\vspace{\fill}

\LARGE Εκτίμηση ιδιοτήτων εδάφους με χρήση\\
\LARGE δισδιάστατων συνελικτικών νευρωνικών δικτύων
\LARGE και τεχνικές βαθειάς μηχανικής μάθησης

\vspace{\fill}

\Large Διπλωματική Εργασία\\
\Large του\\
\Large Χαράλαμπου Παπαδιάκου

\vspace{\fill}
\raggedright

\begin{tabular}{ll}
\textbf{Επιβλέπων:} & Ιωάννης Θεοχάρης\\
 & Καθηγητής Α.Π.Θ.\\
\end{tabular}

\centering
\vspace{\fill}
\today

\end{titlepage}

\begin{abstract}
Η τεχνητή νοημοσύνη διεισδύει ολοένα και περισσότερο στην καθημερινότητα μας, υποσχόμενη να διευκολύνει το σύνολο της ανθρωπότητας μέσω της απαλλαγής της από χρονοβόρες και απαιτητικές σε ανθρώπινο δυναμικό διαδικασίες ή την απλοποίηση τους. Ένα πεδίο εφαρμογής της τεχνητής νοημοσύνης το οποίο αναπτύσσεται ραγδαία είναι η φασματοσκοπία εδάφους, όπου αποσκοπείται να ελαχιστοποιηθεί η προσπάθεια που απαιτείται για την εξαγωγή των χαρακτηριστικών σε διάφορες εδαφικές ύλες. Ένας μηχανισμός που θα μπορούσε να προβλέψει αυτά τα χαρακτηριστικά με ακρίβεια χρησιμοποιώντας μόνο την φασματική υπογραφή του συγκεκριμένου εδάφους, θα πρόσφερε σίγουρα μια αξιοσημείωτη εναλλακτική, σε σχέση με την απαραίτητη μακροσκελή διαδικασία που απαιτείται κατά την αναλυτική εργαστηριακή μέτρηση των ιδιοτήτων του.\\

Στην παρούσα διπλωματική εργασία γίνεται μια προσπάθεια εκτίμησης χημικών και άλλων ιδιοτήτων του εδάφους ως περαιτέρω ανάλυση πάνω σε ήδη υπάρχουσες υλοποιήσεις για την διεκπεραίωσή του παραπάνω σκοπού με τη χρήση συνελικτικών νευρωνικών δικτύων 2 διαστάσεων.\\

Λόγω της πολυπλοκότητας που φέρει η εφαρμογή μιας αρχιτεκτονικής των συγκεκριμένων δικτύων, γίνεται μια προσπάθεια αύξησης της απόδοσης του συνολικού συστήματος πρόβλεψης, μέσω της ελαχιστοποίησης των παραμέτρων που ορίζουν το μέγεθος του νευρωνικού δικτύου χωρίς ενώ ταυτοχρόνως διατηρείται σε επιθυμητά επίπεδα η τελική επίδοση του.\\

Η υλοποίηση γίνεται στη γλώσσα \tl{Python} με τη χρήση των βιβλιοθηκών \tl{Tensorflow - Keras}. Τα πειράματα που πραγματοποιήθηκαν για την εξαγωγή των αποτελεσμάτων έλαβαν χώρο στην ιδρυματική συστοιχία του Α.Π.Θ.
\todo{Συνήθως δε βάζουμε \tl{citations} εδώ}
\end{abstract}

\selectlanguage{english}
\begin{abstract}
Artificial Intelligence is becoming more and more a part of our daily life, offering convenient alternatives to time consuming and human resource demanding processes, or providing a helping hand for their conduction. An area of appliance for artificial intelligence that is rapidly growing is soil spectroscopy, aiming to minimize the work that is required to determine the characteristics of variant soil matters. A mechanism that could predict these soil characteristics accurately using only the spectral signature of specific soil is certain to provide a considerable option over tremendous amount of work\\

This diploma thesis regards a procedure of soil property prediction as an extend of existing implementations on the matter, using two-dimensional convolutional neural networks.\\

Due to the the complexity that arises with the application of models of such architecture, it is attempted to increase the efficiency of the whole prediction model, by minimizing parameters that affect the size of the neural network while maintaining it's performance as high as implementable.\\

The implementation is done in Python using the libraries Tensorflow\cite{tensorflow} - Keras\cite{keras}. The conducted tests are made with the help of the Aristotle University of Thessaloniki's High Performance Computing cluster.~\cite{hpcauth}
\end{abstract}

\thispagestyle{empty}

\selectlanguage{greek}

\section*{Ευχαριστίες}
Θα ήθελα να ευχαριστήσω τον υπεύθυνο καθηγητή μου Κ. Θεοχάρη Ιωάννη καθώς και τον υποψήφιο διδάκτορα Νίκο Τσακιρίδη για την πολύτιμη βοήθεια τους στην περάτωση της παρούσας διπλωματικής εργασίας. Επίσης θα ήθελα να ευχαριστήσω την οικογένεια και τους φίλους μου για την υποστήριξη τους στο διάστημα αυτό της εκπόνησης της διπλωματικής εργασίας αλλά και της πορείας μου ως φοιτητής.
\thispagestyle{empty}

\clearpage
