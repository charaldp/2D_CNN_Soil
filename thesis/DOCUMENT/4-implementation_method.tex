\chapter{Μέθοδος υλοποίησης}
\label{ch:implementation_method}

\section{Προτεινόμενη υλοποίηση}
Η υλοποίηση που προτείνεται στο άρθρο του \tl{Padarian} αναφέρει πως δεν τροποποιεί την είσοδο στα αρχικά δεδομένα. Αυτό έχει ως αποτέλεσμα η εικόνα που προκύπτει στην είσοδο του συνελικτικού νευρωνικού δικτύου να είναι σχετικά μεγάλη. Επίσης η αρχιτεκτονική περιλαμβάνει ένα μεγάλο πλήθος επιπέδων. Ο συνδυασμός των 2 παραπάνω καθιστά την εκπαίδευση του μοντέλου ιδιαίτερα κοστοβόρα λόγω του μεγάλου πλήθους παραμέτρων.

Η υλοποίηση θα μπορούσε να τροποποιηθεί...

\section{Υποδειγματοληψία φάσματος εισόδου --- αποτελεσματικότητα με ελαχιστοποίηση μεγέθους μοντέλου}

\section{Μεθόδοι επίβλεψης της εκπαίδευσης του μοντέλου}

\subsection{Απόκλιση του σφάλματος \tl{validation} κατά την εκπαίδευση}

\subsection{Επαναρχικοποίηση εκπαίδευσης σε περίπτωση ακατάλληλων βαρών}

\subsection{\tl{Checkpoint} μοντέλου κατά την εκπαίδευση}
Ανάκτηση βέλτιστων παραμέτρων με βάση τον σφάλμα \tl{validation}

\subsection{Παραμέτροι εκπαίδευσης}
Κατάλληλος αριθμός εποχών, \tl{Batch-Size}

\section{Τροποποίηση αρχιτεκτονικής προτεινόμενου μοντέλου}

\section{Μοντέλο μιας εισόδου - μιας εξόδου}

\section{Μοντέλο πολλαπλών εξόδων}

\section{Μοντέλο πολλαπλών εισόδων -- εξόδων}

\section{Μοντέλο πολλαπλών εισόδων -- μιας εξόδου}

\section{Υλοποίηση εξαγωγή \tl{Absorbances--Savintzky Golay} 1η παράγωγος από \tl{Reflectances}}

\section{Κανονικοποίηση δεδομένων εισόδου -- εξόδου}

\subsection{Κανονικοποίηση εισόδου}

\subsection{Κανονικοποίηση εξόδου}
