\chapter{Πειραματικά Αποτελέσματα}
\label{ch:experiments_results}

\section{Επίδοση μοντέλων με χρήση δεδομένων εισόδου \tl{Abs-SG1 -- Reflectance spectrograms}}
Κατά την χρήση των \tl{Reflectances spectrograms} παρατηρήθε η βέλτιστη επίδοση για τις εξόδους \tl{Organic Carbon, Nitrogen, Cation Exchange Capacity, Clay, Sand} πλήν του \tl{pH in $H_2O$} οπου παρατηρήθηκε καλύτερη επίδοσή με τη χρήση \tl{Abs-SG1 spectrograms} ως είσοδο.

\section{Επίδοση μοντέλου με χρήση πολλαπλών εισόδων}
Έπειτα από την δοκιμή της επίδοσης των 2 εισόδων έγινε μια απόπειρα για την χρήση των 2 εισόδων σε ένα μοντέλο με σκοπό την εξαγωγή καλύτερων επιδόσεων. Το αποτέλεσμα πράγματα δείχνει μια βελτίωση της επίδοσης του μοντέλου πολλαπλών εισόδων, με την τήρηση του βέλτιστου αποτελέσματος για την κάθε είσοδο, ενώ ταυτόχρονα παρατηρήθηκε μια συνολική βελτίωση της επίδοσης του νευρωνικού δικτύου.

\section{Επίδοση μοντέλων πολλαπλών εξόδων -- μιας εξόδου}
Η απόπειρα πρόβλεψης πολλαπλών ιδιοτήτων με τη χρήση ενός μοντέλου, απαιτεί μια πολυπλοκότερη διαδικασία εκπαίδευσης λόγω του μεγέθους του νέου μοντέλου. Το αποτέλεσμα ωστόσο είναι ικανοποιητικό. Παρόλο που παρατηρείται μια ελαφρώς χειρότερη επίδοση για όλες τις εξόδους, η προβλέψεις του μοντέλου είναι κοντά σε αυτές των μοντέλων πρόβλεψης μιας ιδιότητας εδάφους, πράγμα που δείχνει πως η διαδικασία εκπαίδευσης ενός ξεχωριστού μοντέλου για κάθε ιδιότητα είναι σπάταλη.

\begin{figure}[htbp]
    \begin{subfigure}{0.5\textwidth}
        \includesvg[width=1\linewidth]{RESULTS/OC}
        \caption{Caption 1}
        \label{fig:subim1}
    \end{subfigure}
    \begin{subfigure}{0.5\textwidth}
        \includesvg[width=1\linewidth]{RESULTS/CEC}
        \caption{Caption 2}
        \label{fig:subim2}
    \end{subfigure}
    \begin{subfigure}{0.5\textwidth}
        \includesvg[width=1\linewidth]{RESULTS/Clay}
        \caption{Caption 1}
        \label{fig:subim3}
    \end{subfigure}
    \begin{subfigure}{0.5\textwidth}
        \includesvg[width=1\linewidth]{RESULTS/Sand}
        \caption{Caption 1}
        \label{fig:subim4}
    \end{subfigure}
    \begin{subfigure}{0.5\textwidth}
        \includesvg[width=1\linewidth]{RESULTS/pH}
        \caption{Caption 1}
        \label{fig:subim5}
    \end{subfigure}
    \begin{subfigure}{0.5\textwidth}
        \includesvg[width=1\linewidth]{RESULTS/N}
        \caption{Caption 1}
        \label{fig:subim6}
    \end{subfigure}
    \caption{Σύγκριση μοντέλων για τον οργανικό άνθρακα με βάση την μετρική \tl{RMSE}}
\end{figure}

\section{Επιρροή του λόγου οριζόντιας προς κατακόρυφης διάστασης --- ανώτατο όριο αύξησης}


\section{Μείωση πλήθους \tl{Layers}}

\section{Επιρροή χρήσης \tl{Max Pooling}}
Παρατηρήθηκε πως η βέλτιστη επίδοση του μοντέλου επιτυγχάνεται με χρήση ενός επιπέδου \tl{Max Pooling} 