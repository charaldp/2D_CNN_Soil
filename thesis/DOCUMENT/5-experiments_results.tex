\chapter{Πειράματα και Αποτελέσματα}
\label{ch:experiments_and_results}

Τα αποτελέσματα εξήχθησαν με χρήση της ιδρυματικής συστοιχίας του Α.Π.Θ. \cite{hpcauth} όπου η διαδικασία της εκπαίδευσης επιταγχύνεται με μια κάρτα γραφικών $NVIDIA~Tesla~P100$. Επίσης έγινε χρήση του προσωπικού υπολογιστή για κάποια από τα τελευταία πειράματα με τη χρήση κάρτας γραφικών $NVIDIA~RTX-2060$

\section{Επίδοση μοντέλων με χρήση μετασχηματισμών της εισόδου \tl{Abs-SG1 -- Reflectance spectrograms}}
Η χρήση των σπεκτρογραμμάτων που προκύπτουν από την ανακλαστικότητα \tl{Reflectance} του εδάφους παρατηρείται και στην προτεινόμενη υλοποίηση, ωστόσο έπειτα από δοκιμές χρήσης διαφόρων μετασχηματισμών της εισόδου στο μοντέλο παρατηρήθηκε πως η χρήση της 1ης παραγώγου του μετασχηματισμού \tl{Savitzky-Golay} έχει καλύτερες επιδόσεις για την ιδιότητα του \tl{pH} στο νερό, ενώ ελαφρώς χειρότερη επίδοση από το μοντέλο που χρησιμοποιεί τα σπεκτρογράμματα από ανακλαστικότητα για τις υπόλοιπες ιδιότητες \tl{Organic Carbon, Nitrogen, Cation Exchange Capacity, Clay, Sand}.

\section{Επίδοση μοντέλου με χρήση πολλαπλών εισόδων}
Έπειτα από την δοκιμή της επίδοσης των 2 εισόδων έγινε μια απόπειρα για την χρήση των 2 εισόδων σε ένα μοντέλο με σκοπό την εξαγωγή καλύτερων επιδόσεων. Το αποτέλεσμα πράγματα δείχνει μια βελτίωση της επίδοσης του μοντέλου πολλαπλών εισόδων, με την τήρηση του βέλτιστου αποτελέσματος για την κάθε είσοδο, ενώ ταυτόχρονα παρατηρήθηκε μια συνολική βελτίωση της επίδοσης του νευρωνικού δικτύου.

\section{Επίδοση μοντέλων πολλαπλών εξόδων -- μιας εξόδου}
Η απόπειρα πρόβλεψης πολλαπλών ιδιοτήτων με τη χρήση ενός μοντέλου, απαιτεί μια πολυπλοκότερη διαδικασία εκπαίδευσης λόγω του μεγέθους του νέου μοντέλου. Το αποτέλεσμα ωστόσο είναι ικανοποιητικό. Παρόλο που παρατηρείται μια ελαφρώς χειρότερη επίδοση για όλες τις εξόδους, η προβλέψεις του μοντέλου είναι κοντά σε αυτές των μοντέλων πρόβλεψης μιας ιδιότητας εδάφους, πράγμα που δείχνει πως η διαδικασία εκπαίδευσης ενός ξεχωριστού μοντέλου για κάθε ιδιότητα είναι σπάταλη.

\begin{figure}[htbp]
    \begin{subfigure}{0.5\textwidth}
        \includesvg[width=1\linewidth]{RESULTS/OC}
        \caption{Σύγκριση μοντέλων για την ιδιότητα του οργανικού άνθρακα}
        \label{fig:subim1}
    \end{subfigure}
    \begin{subfigure}{0.5\textwidth}
        \includesvg[width=1\linewidth]{RESULTS/CEC}
        \caption{Σύγκριση μοντέλων για την ιδιότητα της ικανότητας ανταλλαγής κατιόντων}
        \label{fig:subim2}
    \end{subfigure}
    \begin{subfigure}{0.5\textwidth}
        \includesvg[width=1\linewidth]{RESULTS/Clay}
        \caption{Σύγκριση μοντέλων για την ιδιότητα την περιεκτικότητα σε άργιλο}
        \label{fig:subim3}
    \end{subfigure}
    \begin{subfigure}{0.5\textwidth}
        \includesvg[width=1\linewidth]{RESULTS/Sand}
        \caption{Σύγκριση μοντέλων για την ιδιότητα την περιεκτικότητα σε άμμο}
        \label{fig:subim4}
    \end{subfigure}
    \begin{subfigure}{0.5\textwidth}
        \includesvg[width=1\linewidth]{RESULTS/pH}
        \caption{Σύγκριση μοντέλων για την ιδιότητα την περιεκτικότητα του \tl{pH} στο νερό}
        \label{fig:subim5}
    \end{subfigure}
    \begin{subfigure}{0.5\textwidth}
        \includesvg[width=1\linewidth]{RESULTS/N}
        \caption{Σύγκριση μοντέλων για την ιδιότητα της περιεκτικότητας του αζώτου στο έδαφος}
        \label{fig:subim6}
    \end{subfigure}
    \caption{Σύγκριση των προτεινόμενων μοντέλων SI-SO μιας εισόδου --- μιας εξόδου και SI-MO μιας εισόδου --- πολλαπλών εξόδων με τα πειραματικά με ελαχιστοποιημένη αρχιτεκτονική MI-MO πολλαπλών εισόδων --- πολλαπλών εξόδων, MI-SO πολλαπλών εισόδων --- μιας εξόδου, με βάση την μετρική \tl{RMSE}}
\end{figure}

\section{Επιρροή του λόγου οριζόντιας προς κατακόρυφης διάστασης --- ανώτατο όριο αύξησης}


\section{Μείωση πλήθους \tl{Layers}}
Ένας από τους λόγους για τους οποίους η αρχιτεκτονική του μοντέλου των \tl{Padarian J. et al} θεωρείται μη αποτελεσματική είναι ότι χρησιμοποιεί πολλά επίπεδα 

\section{Επιρροή χρήσης \tl{Max Pooling}}
Όπως αναφέρθηκε στην θεωρητική ανάλυση των δισδιάστατων νευρωνικών δικτύων τα επίπεδα συγκέντρωσης είναι από τα βασικά της συγκεκριμένης κατηγορίας μοντέλων. Ωστόσο εξετάστηκε η επίδραση της μείωσης του αριθμού των επιπέδων συγκέντρωσης καθώς σε ορισμένες αρχιτεκτονικές του δοκιμάστηκαν έχουν μειωμένο αριθμό συνελικτικών επιπέδων. Κατά την δοκιμή της χρήσης διαφόρων αριθμών επιπέδων συγκέντρωσης παρατηρήθηκε πως η αφαίρεση όλων των επιπέδων συγκέντρωσης είχε ως αποτέλεσμα την σημαντική μείωση της επίδοσης του μοντέλου. Τελικά διαπιστώθηκε πως η βέλτιστη επίδοση του μοντέλου επιτυγχάνεται με χρήση ενός επιπέδου συγκέντρωσης

Πίνακες επίδοσης μοντέλων με διάφορες χρήσεις επιπέδων συγκέντρωσης

\section{Οπτικοποίηση συνελικτικών φίλτρων του μοντέλου}
